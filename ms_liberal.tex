\documentclass[letterpaper]{article}
\usepackage[utf8]{inputenc}
\usepackage{authblk}
\usepackage{hyperref}
\usepackage[usenames,dvipsnames,svgnames,table]{xcolor}

\definecolor{myColor}{HTML}{006699}

\hypersetup {
  colorlinks = true, linkcolor = myColor, citecolor = myColor, urlcolor = myColor,
  pdfauthor = {Desjardins-Proulx, Philippe},
}

\renewcommand\Affilfont{\itshape\small}
\setcounter{section}{-1} % Sections too start at 0 :P

\begin{document}

\title{Restrictive Licenses for Code and Data Hinder Science}
\author[0,1,2]{Philippe Desjardins-Proulx}
\affil[0]{email: \href{mailto:philippe.d.proulx@gmail.com}{philippe.d.proulx@gmail.com}}
\affil[1]{Quebec Center for Biodiversity Science, Canada.}
\affil[2]{Universit\'e du Qu\'ebec, Canada.}
\date{\today}
\maketitle

\section{Introduction}

The rapid increase in data, along with more accessible tools for analysis, is
paving the way for data-intensive science \cite{hey09}. On a practical side, it
means more data-sets, with many built with many smaller data-sets, and more
code to analyze the data-sets. It also means scientists have to select
licenses for their code and data \cite{mor12}. In this short opinion, we
argue that complex restrictive licenses should be avoided for code and data.

We define restrictive licenses as those that put many restrictions on code/data
reuse, and oppose them to permissive licenses with almost no restrictions. For
data, the CC0 and CC-BY are permissive, unlike the creative commons with the NC
(non-commercial) or SA (share-alike) clauses. For code, the MIT/BSD/Apache
licenses are generally considered permissive and liberal, while the GPL is a
copyleft license that puts many restrictions on code reuse (see BOX 1). The
temptation to add restrictions can be strong. After all, programmers spend
hours building high-quality software, and field scientists spend equally long
hours collecting data. Why would you allow your code to be used without
restrictions?

The question of licensing invariably leads to complex philosophical questions.
Does \emph{open} means to let others do as they like with data and code? Or,
does \emph{open} means we need to restrict usage so the data/code would never
be used in a closed environment? Instead of rehashing this debate, we will
focus on two practical problems with restrictive licenses. (1) Restrictive
licenses cause problems downstream, something we cannot afford in a world of
Big Data and massive data-sets. (2) Restrictive licenses are mired with legal
ambiguities. They are complex to understand, and complex to apply across
countries with different legal systems.

\section{Box 1: License Madness 101}

\begin{table}
  \centering
  \caption{\bf{What code can you use given your artifact's license?}}
  \begin{tabular}{|l|ccc|}
  \hline
  Artifact              & Permissive & Copyleft & Proprietary \\
  \hline
  Permissive            & Yes        & No       & No          \\
  Copyleft              & Yes        & Yes      & No          \\
  Proprietary           & Yes        & No       & Yes         \\
  \hline
  \end{tabular}
  \label{table:codelicenses}
\end{table}

Table \ref{table:codelicenses} shows what code you use given the license you
want to have for your artifact. The 'artifact' is what you end up executing.
It is a suble but important point: you can license your code under a permissive
license and use copyleft code for your project, but when the code is compiled
to be executed it has to be distributed under the GPL license. This table is a
simplification in other ways. For example, GPL code can link to proprietary
code if it is a "system library". Also, while the basic idea of the GPL license is
that you need to distribute code with artifacts, it is possible for a company
to hold a large GPL codebase without ever giving the code if they don't need to
distribute the software directly (which is common for companies operating on
the cloud). There are many other subtleties, and most of them have not been
tested in court.

\section{Downstream problems}

One of the most exiting prospect of Big Data is the possibility of analyzing
and learning from large comprehensive data-sets. It has been argued that
large data-sets are ``unreasonably effective'', that simple approaches often
work surprisingly well when provided with enough data \cite{hal09}. Unfortunately,
restriction accumulate. To be truly CC-BY, a data-set has to use only CC0 and
CC-BY data-sets. Once data-set with more restrictions are added, the data-set
cannot be used without these restrictions, unless of course the restrictive
data-sets are removed.

Source code face similar issues. Roughly half of the open source code is
under the copyleft GPL license, and thus cannot be used with permissive
code. This has led to strange situations, such as motivating the LLVM project
to write from scratch a new C++ standard library instead of collaborating
with the existing GNU project.

% I looked at a bunch of trending repositories on github and 100% of them were under a liberal
% license. I'll get more data, but it seems there is a pretty strong trend toward liberal
% licenses. 
% -- PhDP

\section{The case for simplicity}

Most of us would rather do science and focus on solving problems than discuss
legal issues.  And licenses can be really complex, potentially different
countries having different interpretations. Recently, a German court judged
that the creative commons with non-commercial clause could not be used by
a non-commercial entity. %(REF + mode details)

Liberal licenses are simple. Code under the MIT or BSD licenses can be used by
anyone. Restrictive open licenses are often less accessible than proprietary
code/data: if you want to build a program with your MIT code, you cannot use
the GNU Scientific Library without adopting the GPL for your executable, but
you can use the proprietary Intel Math Kernel.

Manuscripts can be written openly with git \cite{ram13}.

\section{Recommended references}
A list of recommended reference from twitter:
\begin{itemize}
\item http://www.astrobetter.com/the-whys-and-hows-of-licensing-scientific-code/
\item http://t.co/eVTdJgmmbx
\item http://nipy.sourceforge.net/nipy/stable/faq/johns_bsd_pitch.html
\item http://web.stanford.edu/~vcs/papers/LFRSR12012008.pdf
\item http://blog.datadryad.org/2011/10/05/why-does-dryad-use-cc0/

\end{itemize}

\section{Acknowledgements}

Carter T. Schonwald for a clarification on the GPL license.

\bibliography{refs}
\bibliographystyle{plain}

\end{document}

